\documentclass[fleqn]{article}
\usepackage[top=30truemm,bottom=25truemm,left=25truemm,right=25truemm]{geometry}
\usepackage{color}
\usepackage{amsmath, amssymb}

\usepackage[dvipdfmx,hiresbb]{graphicx}
\usepackage{float}
\usepackage{here}
\usepackage[english]{babel}
\usepackage{bm}

\usepackage{latexsym}

\title{Non-Bayesian Time-Varying Vector Autoregressive Model}
\date{\today}
\author{Ryoga Kobayashi}
\begin{document}

\maketitle

\section{Model}

\begin{align*}
    \bm y_t = \bm \nu + \bm \Phi^{(1)}_{t} \bm y_{t-1} + \cdots + \bm \Phi^{(p)}_{t} \bm y_{t-p} + \bm \epsilon_t, \qquad t = p+1, \dots, T,
\end{align*}
where
\begin{align*}
    & \underbrace{\bm y_t}_{d \times 1} = \begin{pmatrix}
        y_{1, t} \\
        \vdots \\
        y_{d, t}
    \end{pmatrix},
    \qquad
    \underbrace{\bm \nu}_{d \times 1} = \begin{pmatrix}
        \nu_1 \\
        \vdots \\
        \nu_d
    \end{pmatrix},
    \qquad
    \underbrace{\bm \epsilon_t}_{d \times 1} = \begin{pmatrix}
        \epsilon_{1, t} \\
        \vdots \\
        \epsilon_{d, t}
    \end{pmatrix},
    \qquad
    \underbrace{\bm \Phi^{(l)}_t}_{d \times d} = \begin{pmatrix}
        \phi^{(l)}_{11, t} & \cdots & \phi^{(l)}_{1d, t} \\
        \vdots & \ddots & \vdots \\
        \phi^{(l)}_{d1, t} & \cdots & \phi^{(l)}_{dd, t}
    \end{pmatrix},
    \qquad
    l = 1, \dots, p.
\end{align*}

We can notate as
\begin{align*}
    \begin{pmatrix}
        y_{1, t} \\
        \vdots \\
        y_{d, t}
    \end{pmatrix}
    &=
    \begin{pmatrix}
        \nu_1 \\
        \vdots \\
        \nu_d
    \end{pmatrix}
    +
    \begin{pmatrix}
        \phi^{(1)}_{11, t} & \cdots & \phi^{(1)}_{1d, t} \\
        \vdots & \ddots & \vdots \\
        \phi^{(1)}_{d1, t} & \cdots & \phi^{(1)}_{dd, t}
    \end{pmatrix}
    \begin{pmatrix}
        y_{1, t-1} \\
        \vdots \\
        y_{d, t-1}
    \end{pmatrix}
    + \cdots +
    \begin{pmatrix}
        \phi^{(p)}_{11, t} & \cdots & \phi^{(p)}_{1d, t} \\
        \vdots & \ddots & \vdots \\
        \phi^{(p)}_{d1, t} & \cdots & \phi^{(p)}_{dd, t}
    \end{pmatrix}
    \begin{pmatrix}
        y_{1, t-p} \\
        \vdots \\
        y_{d, t-p}
    \end{pmatrix}
    +
    \begin{pmatrix}
        \epsilon_{1, t} \\
        \vdots \\
        \epsilon_{d, t}
    \end{pmatrix} \\
    & =
    \begin{pmatrix}
        \nu_1 \\
        \vdots \\
        \nu_d
    \end{pmatrix}
    +
    \begin{pmatrix}
        \phi^{(1)}_{11, t} & \cdots & \phi^{(1)}_{1d, t}
        & \cdots &
        \phi^{(p)}_{11, t} & \cdots & \phi^{(p)}_{1d, t} \\
        \vdots & \ddots & \vdots
        & \cdots &
        \vdots & \ddots & \vdots \\
        \phi^{(1)}_{d1, t} & \cdots & \phi^{(1)}_{dd, t}
        & \cdots &
        \phi^{(1)}_{d1, t} & \cdots & \phi^{(1)}_{dd, t} \\
    \end{pmatrix}
    \begin{pmatrix}
        y_{1, t-1} \\
        \vdots \\
        y_{d, t-p} \\
        \vdots \\
        y_{1, t-1} \\
        \vdots \\
        y_{d, t-p} \\
    \end{pmatrix}
    +
    \begin{pmatrix}
        \epsilon_{1, t} \\
        \vdots \\
        \epsilon_{d, t}
    \end{pmatrix}.
\end{align*}

Thus, we can write
\begin{align*}
    \bm y_t = \bm \nu + \bm \Phi_t \bm Z_t + \bm \epsilon_t,
\end{align*}
where
\begin{align*}
    \underbrace{\bm \Phi_t}_{d \times dp} = \begin{pmatrix}
        \phi^{(1)}_{11, t} & \cdots & \phi^{(1)}_{1d, t}
        & \cdots &
        \phi^{(p)}_{11, t} & \cdots & \phi^{(p)}_{1d, t} \\
        \vdots & \ddots & \vdots
        & \cdots &
        \vdots & \ddots & \vdots \\
        \phi^{(1)}_{d1, t} & \cdots & \phi^{(1)}_{dd, t}
        & \cdots &
        \phi^{(1)}_{d1, t} & \cdots & \phi^{(1)}_{dd, t} \\
    \end{pmatrix},
    \qquad
    \underbrace{\bm Z_t}_{dp \times 1} = \begin{pmatrix}
        \bm y_{t-1} \\
        \vdots \\
        \bm y_{t-p}
    \end{pmatrix}
    = \begin{pmatrix}
        y_{1, t-1} \\
        \vdots \\
        y_{d, t-1} \\
        \vdots \\
        y_{1, t-p} \\
        \vdots \\
        y_{d, t-p}
    \end{pmatrix}.
\end{align*}

we can calculate
\begin{align*}
    \begin{pmatrix}
        y_{1, t} \\
        \vdots \\
        y_{d, t}
    \end{pmatrix}
    & =
    \begin{pmatrix}
        \nu_1 \\
        \vdots \\
        \nu_d
    \end{pmatrix}
    +
    \begin{pmatrix}
        \phi^{(1)}_{11, t} & \cdots & \phi^{(1)}_{1d, t}
        & \cdots &
        \phi^{(p)}_{11, t} & \cdots & \phi^{(p)}_{1d, t} \\
        \vdots & \ddots & \vdots
        & \cdots &
        \vdots & \ddots & \vdots \\
        \phi^{(1)}_{d1, t} & \cdots & \phi^{(1)}_{dd, t}
        & \cdots &
        \phi^{(1)}_{d1, t} & \cdots & \phi^{(1)}_{dd, t} \\
    \end{pmatrix}
    \begin{pmatrix}
        y_{1, t-1} \\
        \vdots \\
        y_{d, t-p} \\
        \vdots \\
        y_{1, t-1} \\
        \vdots \\
        y_{d, t-p} \\
    \end{pmatrix}
    +
    \begin{pmatrix}
        \epsilon_{1, t} \\
        \vdots \\
        \epsilon_{d, t}
    \end{pmatrix} \\
    &= \begin{pmatrix}
        1 & & \\
        & \ddots & \\
        & & 1
    \end{pmatrix}
    \begin{pmatrix}
        \nu_1 \\
        \vdots \\
        \nu_d
    \end{pmatrix}
    + \begin{pmatrix}
        y_{1, t-1} &        &            & \cdots & y_{d, t-1} &        &            \\
                   & \ddots &            & \cdots &            & \ddots &            \\
                   &        & y_{1, t-p} & \cdots &            &        & y_{d, t-p}
    \end{pmatrix}
    \begin{pmatrix}
        \nu_1 \\
        \vdots \\
        \nu_d \\
        \phi^{(1)}_{11, t} \\
        \vdots \\
        \phi^{(1)}_{d1, t} \\
        \vdots \\
        \phi^{(p)}_{d1, t} \\
        \vdots \\
        \phi^{(p)}_{dd, t}
    \end{pmatrix}
    +
    \begin{pmatrix}
        \epsilon_{1, t} \\
        \vdots \\
        \epsilon_{d, t}
    \end{pmatrix}
\end{align*}

Therefore we can notate
\begin{align*}
    \bm y_t = \begin{pmatrix}
        \bm I_d & \bm Z_t' \otimes \bm I_d
    \end{pmatrix}
    \begin{pmatrix}
        \bm \nu \\
        \operatorname{vec}(\bm \Phi_t)
    \end{pmatrix}
\end{align*}

We can write
\begin{align*}
    \begin{pmatrix}
        \bm y_{p+1} \\
        \vdots \\
        \bm y_T
    \end{pmatrix}
    =
    \begin{pmatrix}
        \bm I_d & \bm Z_{p+1}' \otimes \bm I_d &        &                                 \\
        \vdots  &                                 & \ddots &                              \\
        \bm I_d &                                 &        & \bm Z_{T}' \otimes \bm I_d
    \end{pmatrix}
    \begin{pmatrix}
        \bm \nu \\
        \operatorname{vec}(\bm \Phi_{p+1}) \\
        \vdots \\
        \operatorname{vec}(\bm \Phi_T)
    \end{pmatrix}
    +
    \begin{pmatrix}
        \bm \epsilon_{p+1} \\
        \vdots \\
        \bm \epsilon_T
    \end{pmatrix}
\end{align*}

Our model can be rewritten as
\begin{align*}
    \bm Y = \bm Z \bm \beta + \bm \epsilon
\end{align*}
where
\begin{align*}
    \underbrace{\bm Y}_{d(T-p) \times 1} = \begin{pmatrix}
        \bm y_{p+1} \\
        \vdots \\
        \bm y_T
    \end{pmatrix},
    \qquad
    \underbrace{\bm Z}_{d(T-p) \times (d + d^2p(T-p))} =
    \begin{pmatrix}
        \bm I_d & \bm Z_{p+1}' \otimes \bm I_d &        &                            \\
        \vdots  &                              & \ddots &                            \\
        \bm I_d &                              &        & \bm Z_{T}' \otimes \bm I_d
    \end{pmatrix}, \\
\end{align*}
and
\begin{align*}
    \underbrace{\bm \beta}_{(d + d^2p(T-p)) \times 1}
    = \begin{pmatrix}
        \bm \nu \\
        \operatorname{vec}(\bm \Phi_{p+1}) \\
        \vdots \\
        \operatorname{vec}(\bm \Phi_{T})
    \end{pmatrix},
    \qquad
    \underbrace{\bm \epsilon}_{d(T-p) \times 1}
    = \begin{pmatrix}
        \bm \epsilon_{p+1} \\
        \vdots \\
        \bm \epsilon_T
    \end{pmatrix}.
\end{align*}

Assume
\begin{align*}
    \phi^{(l)}_{ij, t} = \phi^{(l)}_{ij, t-1} + h^{(l)}_{ij, t},
    \qquad i = 1, \dots, d,
    \qquad j = 1, \dots, d,
    \qquad l = 1, \dots, p,
    \qquad t = p+1, \dots, T.
\end{align*}

Note that
\begin{align*}
    \bm \Phi_t = \bm \Phi_{t-1} + \bm H_t
\end{align*}
where
\begin{align*}
    \underbrace{\bm H_t}_{d \times dp} = \begin{pmatrix}
        h^{(1)}_{11, t} & \cdots & h^{(1)}_{1d, t} & \cdots & h^{(p)}_{11, t} & \cdots & h^{(p)}_{1d, t} \\
        \vdots          & \ddots & \vdots          & \cdots & \vdots          & \ddots & \vdots \\
        h^{(1)}_{d1, t} & \cdots & h^{(1)}_{dd, t} & \cdots & h^{(p)}_{d1, t} & \cdots & h^{(p)}_{dd, t}
    \end{pmatrix}.
\end{align*}

Define
\begin{align*}
    \underbrace{\bm \beta_t}_{d^2p \times 1} = \operatorname{vec}(\bm \Phi_t), \qquad \underbrace{\bm \eta_t}_{d^2p \times 1} = \operatorname{vec}(\bm H_t),
\end{align*}
then
\begin{align*}
    \bm \beta_t = \bm \beta_{t-1} + \bm \eta_t.
\end{align*}

We can calculate
\begin{align*}
    \begin{pmatrix}
        \bm \beta_{p+1} \\
        \vdots \\
        \bm \beta_{T}
    \end{pmatrix}
    &=
    \begin{pmatrix}
        \bm \beta_p \\
        \vdots \\
        \bm \beta_{T-1}
    \end{pmatrix}
    +
    \begin{pmatrix}
        \bm \eta_{p+1} \\
        \vdots \\
        \bm \eta_T
    \end{pmatrix} \\
    \begin{pmatrix}
        \bm 0_{d^2p} \\
        \vdots \\
        \bm 0_{d^2p}
    \end{pmatrix}
    &= \begin{pmatrix}
        \bm \beta_p - \bm \beta_{p+1} \\
        \vdots \\
        \bm \beta_{T-1} - \bm \beta_T
    \end{pmatrix}
    +
    \begin{pmatrix}
        \bm \eta_{p+1} \\
        \vdots \\
        \bm \eta_T
    \end{pmatrix} \\
    \begin{pmatrix}
        - \bm \beta_p \\
        \bm 0_{d^2p} \\
        \vdots \\
        \bm 0_{d^2p}
    \end{pmatrix}
    &= \begin{pmatrix}
        - \bm \beta_{p+1} \\
        \bm \beta_{p+1} - \bm \beta_{p+2} \\
        \vdots \\
        \bm \beta_{T-1} - \beta_{T}
    \end{pmatrix}
    +
    \begin{pmatrix}
        \bm \eta_{p+1} \\
        \vdots \\
        \bm \eta_T
    \end{pmatrix} \\
    &= \begin{pmatrix}
        - \bm I_{d^2p} &                &              &                \\
          \bm I_{d^2p} & - \bm I_{d^2p} &              &                \\
                       & \ddots         & \ddots       &                \\
                       &                & \bm I_{d^2p} & - \bm I_{d^2p}
    \end{pmatrix}
    \begin{pmatrix}
        \bm \beta_{p+1} \\
        \bm \beta_{p+2} \\
        \vdots \\
        \bm \beta_T
    \end{pmatrix}
    +
    \begin{pmatrix}
        \bm \eta_{p+1} \\
        \vdots \\
        \bm \eta_T
    \end{pmatrix} \\
    &= \begin{pmatrix}
        \bm 0_{d^2p} & \cdots & \bm 0_{d^2p} & - \bm I_{d^2p} &                &              &                \\
        \bm 0_{d^2p} & \cdots & \bm 0_{d^2p} &   \bm I_{d^2p} & - \bm I_{d^2p} &              &                \\
        \vdots       & \cdots & \vdots       &                & \ddots         & \ddots       &                \\
        \bm 0_{d^2p} & \cdots & \bm 0_{d^2p} &                &                & \bm I_{d^2p} & - \bm I_{d^2p}
    \end{pmatrix}
    \begin{pmatrix}
        \bm \nu \\
        \bm \beta_{p+1} \\
        \bm \beta_{p+2} \\
        \vdots \\
        \bm \beta_T
    \end{pmatrix}
    +
    \begin{pmatrix}
        \bm \eta_{p+1} \\
        \vdots \\
        \bm \eta_T
    \end{pmatrix}.
\end{align*}

Thus, we can write
\begin{align*}
    \bm \Gamma = \bm W \bm \beta + \bm \eta
\end{align*}
where
\begin{align*}
    \underbrace{\bm \Gamma}_{d^2p(T-p)) \times 1} = \begin{pmatrix}
        - \operatorname{vec}(\bm \Phi_p) \\
        \bm 0_{d^2p} \\
        \vdots \\
        \bm 0_{d^2p}
    \end{pmatrix},
    \qquad
    \underbrace{\bm \eta}_{d^2p(T-p) \times 1} = \begin{pmatrix}
        \bm \eta_{p+1} \\
        \vdots \\
        \bm \eta_T
    \end{pmatrix},
\end{align*}
and
\begin{align*}
    \underbrace{\bm W}_{d^2p(T-p) \times (d+d^2p(T-p))} = \begin{pmatrix}
        \bm 0_{d^2p} & \cdots & \bm 0_{d^2p} & - \bm I_{d^2p} &                &              &                \\
        \bm 0_{d^2p} & \cdots & \bm 0_{d^2p} &   \bm I_{d^2p} & - \bm I_{d^2p} &              &                \\
        \vdots       & \cdots & \vdots       &                & \ddots         & \ddots       &                \\
        \bm 0_{d^2p} & \cdots & \bm 0_{d^2p} &                &                & \bm I_{d^2p} & - \bm I_{d^2p}
    \end{pmatrix}.
\end{align*}

Therefore, with all things considered, our model can be notated as
\begin{align*}
    \bm \psi = \bm \zeta \bm \beta + \bm \xi,
\end{align*}
where
\begin{align*}
    \underbrace{\bm \psi}_{(d(T-p)+d^2p(T-p)) \times 1} = \begin{pmatrix}
        \bm Y \\
        \bm \Gamma
    \end{pmatrix},
    \qquad
    \underbrace{\bm \zeta}_{(d(T-p) + d^2p)(T-p)) \times (d + d^2p(T-p))}
    = \begin{pmatrix}
        \bm Z \\
        \bm W
    \end{pmatrix},
    \qquad
    \underbrace{\bm \xi}_{(d(T-p)+d^2p(T-p)) \times 1}
    = \begin{pmatrix}
        \bm \epsilon \\
        \bm \eta
    \end{pmatrix}.
\end{align*}

\section{Prediction}

\begin{align*}
    \hat{\bm y}_t = E[\bm y_t| \mathcal F_{t-1}] = E[\bm y_t| \bm y_{t-1}, \cdots, \bm y_{t-p}]
\end{align*}

\begin{align*}
    \hat{\bm y}_t = \hat{\bm \nu} + \hat{\bm \Phi}^{(1)}_{t} \bm y_{t-1} + \cdots + \hat{\bm \Phi}^{(p)}_{t} \bm y_{t-p},
\end{align*}
where
\begin{align*}
    & \underbrace{\hat{\bm y}_t}_{d \times 1} = \begin{pmatrix}
        \hat{y}_{1, t} \\
        \vdots \\
        \hat{y}_{d, t}
    \end{pmatrix},
    \qquad
    \underbrace{\bm \nu}_{d \times 1} = \begin{pmatrix}
        \hat{\nu}_1 \\
        \vdots \\
        \hat{\nu}_d
    \end{pmatrix},
    \qquad
    \underbrace{\hat{\bm \Phi}^{(l)}_t}_{d \times d} = \begin{pmatrix}
        \hat{\phi}^{(l)}_{11, t} & \cdots & \hat{\phi}^{(l)}_{1d, t} \\
        \vdots & \ddots & \vdots \\
        \hat{\phi}^{(l)}_{d1, t} & \cdots & \hat{\phi}^{(l)}_{dd, t}
    \end{pmatrix},
    \qquad
    l = 1, \dots, p.
\end{align*}

We can notate as
\begin{align*}
    \begin{pmatrix}
        \hat{y}_{1, t} \\
        \vdots \\
        \hat{y}_{d, t}
    \end{pmatrix}
    &=
    \begin{pmatrix}
        \hat{\nu}_1 \\
        \vdots \\
        \hat{\nu}_d
    \end{pmatrix}
    +
    \begin{pmatrix}
        \hat{\phi}^{(1)}_{11, t} & \cdots & \hat{\phi}^{(1)}_{1d, t} \\
        \vdots & \ddots & \vdots \\
        \hat{\phi}^{(1)}_{d1, t} & \cdots & \hat{\phi}^{(1)}_{dd, t}
    \end{pmatrix}
    \begin{pmatrix}
        y_{1, t-1} \\
        \vdots \\
        y_{d, t-1}
    \end{pmatrix}
    + \cdots +
    \begin{pmatrix}
        \hat{\phi}^{(p)}_{11, t} & \cdots & \hat{\phi}^{(p)}_{1d, t} \\
        \vdots & \ddots & \vdots \\
        \hat{\phi}^{(p)}_{d1, t} & \cdots & \hat{\phi}^{(p)}_{dd, t}
    \end{pmatrix}
    \begin{pmatrix}
        y_{1, t-p} \\
        \vdots \\
        y_{d, t-p}
    \end{pmatrix}\\
    & =
    \begin{pmatrix}
        \nu_1 \\
        \vdots \\
        \nu_d
    \end{pmatrix}
    +
    \begin{pmatrix}
        \phi^{(1)}_{11, t} & \cdots & \phi^{(1)}_{1d, t}
        & \cdots &
        \phi^{(p)}_{11, t} & \cdots & \phi^{(p)}_{1d, t} \\
        \vdots & \ddots & \vdots
        & \cdots &
        \vdots & \ddots & \vdots \\
        \phi^{(1)}_{d1, t} & \cdots & \phi^{(1)}_{dd, t}
        & \cdots &
        \phi^{(1)}_{d1, t} & \cdots & \phi^{(1)}_{dd, t} \\
    \end{pmatrix}
    \begin{pmatrix}
        y_{1, t-1} \\
        \vdots \\
        y_{d, t-p} \\
        \vdots \\
        y_{1, t-1} \\
        \vdots \\
        y_{d, t-p} \\
    \end{pmatrix}.
\end{align*}

Thus we can write
\begin{align*}
    \hat{\bm y}_t = \hat{\bm \nu} + \hat{\bm \Phi}_t \bm Z_t.
\end{align*}

\end{document}
